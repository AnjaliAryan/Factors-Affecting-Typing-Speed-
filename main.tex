\documentclass{beamer}
\usepackage{listings}
\lstset{
%language=C,
frame=single, 
breaklines=true,
columns=fullflexible
}
\usetheme[progressbar=frametitle]{metropolis}
\usepackage{appendixnumberbeamer}

\usepackage{booktabs}
\usepackage[scale=2]{ccicons}
\newenvironment{subcolumns}[1]
 {\valign\bgroup\hsize=#1##\cr}
 {\crcr\egroup}
\newcommand{\nextsubcolumn}{\cr\noalign{\hfill}}
\newcommand{\nextsubfigure}{\vfill}
\usepackage{pgfplots}
\usepgfplotslibrary{dateplot}

\usepackage{xspace}
\newcommand{\themename}{\textbf{\textsc{metropolis}}\xspace}
\usepackage{subcaption}
\usepackage{url}

\usepackage{tikz}
\usetikzlibrary{arrows.meta,positioning}
\usepackage{pgfplots}
\pgfplotsset{compat=1.17}
\usepackage{tkz-fct}
\usepackage{mathrsfs}
\usepackage{txfonts}
\usepackage{tkz-euclide} 
\usetikzlibrary{calc,math}
\usepackage{float}
\newcommand\norm[1]{\left\lVert#1\right\rVert}
\renewcommand{\vec}[1]{\mathbf{#1}}
\providecommand{\pr}[1]{\ensuremath{\Pr\left(#1\right)}}
\usepackage[export]{adjustbox}
\usepackage[utf8]{inputenc}
\usepackage{amsmath}
\newcommand{\SubItem}[1]{
    {\setlength\itemindent{15pt} \item[-] #1}
}
\title{Factors Affecting Typing Speed}
\subtitle{Project Presentation: MA4240 Applied Statistics}
\date{\today}

\author{GROUP-5}
%\institute{Indian Institute of Technology Hyderabad}
\titlegraphic{\hfill\includegraphics[height=1.5cm]{iith logo.png}}

\begin{document}
\metroset{block=fill}
\begin{frame}
\titlepage
\end{frame}
\begin{frame}{Team Members}
\begin{itemize}
    \item Amulya Tallamraju - AI20BTECH11003
    \item Vaibhav Chhabra - AI20BTECH11022
    \item Anita Dash - MA20BTECH11001
    \item Anjali - MA20BTECH11002
    \item Ruthwika Boyapally - MA20BTECH11004
    \item Kunal Nema - MA20BTECH11007
    \item Sparsh Gupta - MA20BTECH11015
    \item Tapishi Kaur - MA20BTECH11017
\end{itemize}
\end{frame}
\begin{frame}{Introduction}
   \begin{block}{Abstract}
     Typing is now one of the essential part of almost every job since it lets you do your work very quickly and efficiently.\newline
     We aim to analyze the typing speed of students in IITH, try drawing conclusions based on various factors and analyze if there is some relation between the factors and the typing speed.
   \end{block} 
\end{frame}
\begin{frame}
\section{Collection of Data}
\end{frame}
\begin{frame}{Objective of the Study}
   \begin{block}{Objective}
     The objective of our study is to ascertain factors affecting the typing speed of students.
   \end{block} 
\end{frame}
\begin{frame}
    \begin{figure}
    \centering
    \includegraphics[width=1\columnwidth]{factors.jpeg}
    \caption{factors considered for the purpose of this study}
    \label{fig:my_label}
\end{figure}
\end{frame}
\begin{frame}
\begin{figure}
    \centering
    \includegraphics[width=1\columnwidth]{factors wpm.jpeg}
    \caption{Is there a correlation between the factors and WPM?}
    \label{fig:my_label}
\end{figure}
\end{frame}
\begin{frame}
   \begin{figure}
    \centering
    \includegraphics[width=1\columnwidth]{factors accuracy.jpeg}
    \caption{Is there a correlation between the factors and Accuracy?}
    \label{fig:my_label}
\end{figure} 
\end{frame}


\begin{frame}
\begin{figure}
    \centering
    \includegraphics[width=1\columnwidth]{variables of interest.jpeg}
    \caption{Variables of Interest}
    \label{fig:my_label}
    \end{figure}
\end{frame}
\begin{frame}
    \begin{block}{}
      We perform an observational study to find out how the above stated factors affect the typing speed of an individual.\\
    \end{block}
\end{frame}
\begin{frame}{Strategy followed for Data Collection}
\begin{itemize}
    \item A google form was circulated to collect data.
    \item To avoid measurement problem in responses, participants were urged to       take a common test.
    \item \href{https://monkeytype.com/}{\beamergotobutton{Link to the Test}}
\end{itemize}
\end{frame}
\begin{frame}{Strategy followed for Data Collection}
    \begin{alertblock}{A Potential Issue}
      The survey we conducted is an example of convenience sampling. Hence, the data frame generated would be biased
    \end{alertblock}
\end{frame}
\begin{frame}{Strategy followed for Data Collection}
    \begin{block}{Definition}
      Simple random sampling is defined as a sampling technique where every item in the population has an even chance and likelihood of being selected in the sample.
    \end{block}
\end{frame}
\begin{frame}
    \begin{figure}
    \centering
    \includegraphics[width=0.9\columnwidth]{strata.jpeg}
    \caption{Simple random sampling of subgroups}
    \label{fig:my_label}
\end{figure} 
\end{frame}
\begin{frame}
    \begin{block}{Histogram Test}
     \begin{figure}
     \centering
     \begin{subfigure}[b]{0.475\textwidth}
         \centering
         \includegraphics[width=\textwidth]{WPM-frequency.png}
         \label{fig:y equals x}
     \end{subfigure}
     \hfill
     \begin{subfigure}[b]{0.475\textwidth}
         \centering
         \includegraphics[width=\textwidth]{Acc-frequency.png}
         \label{fig:three sin x}
     \end{subfigure}
     \caption{The graphs show that the data for wpm is right skewed and the accuracy is left skewed}
        \label{fig:three graphs}
\end{figure}
     \end{block}
\end{frame}
\begin{frame}
    \begin{block}{Q-Q Plot Test}
     \begin{figure}
     \centering
     \begin{subfigure}[b]{0.475\textwidth}
         \centering
         \includegraphics[width=\textwidth]{qq_wpm.png}
         \label{fig:y equals x}
     \end{subfigure}
     \hfill
     \begin{subfigure}[b]{0.475\textwidth}
         \centering
         \includegraphics[width=\textwidth]{qq_accu.png}
         \label{fig:three sin x}
     \end{subfigure}
     \caption{Certain number of points are away from the line}
        \label{fig:three graphs}
\end{figure}
    \end{block}
\end{frame}
\begin{frame}
\begin{block}{p-value test for normality}
   Hypothesis Testing with 0.05 level of significance
   \hyperlink{Hypothesis Testing}{\beamerbutton{More details}}
   \\Null Hypothesis: Data is Normally Distributed \\
   Alternative Hypothesis: Data is not Normally Distributed\\
   \begin{figure}[htp]
\includegraphics[width=\textwidth]{Null Hypothesis.jpeg}

\end{figure}
 
\end{block}
\end{frame}

\begin{frame}
   \begin{alertblock}{Issue}
     Our data is not normally distributed\\
     How do we go ahead with our study?
   \end{alertblock}
   \begin{block}{Central Limit Theorem}
    The Central Limit Theorem states that the sampling distribution of the sample means approaches a normal distribution as the sample size gets larger � no matter what the shape of the population distribution. This fact holds especially true for sample sizes over 30.
   \end{block}
\end{frame} 
\begin{frame}{Sampling distribution for $n=30$: WPM}
\begin{figure}[htp]
\includegraphics[width=\textwidth]{WPM Sampling Distribution.jpeg}

\end{figure}
    Upload graphs for sampling distribution WPM
\end{frame}
\begin{frame}{Sampling distribution for $n=30$: Accuracy}
\begin{figure}[htp]
\includegraphics[width=\textwidth]{Acc Sampling Distribution.jpeg}

\end{figure}
    Upload graphs for sampling distribution Accuracy
\end{frame}
\begin{frame}{Sample}
    \begin{figure}
    \centering
    \includegraphics[width=1\columnwidth]{Data Sampled Numbers.jpeg}
    \caption{Data Collected and Data Sampled}
    \label{fig:my_label}
\end{figure} 
\end{frame}
\begin{frame}
 \section[Summarizing and Visualizing the Sample Data]{Collection of Data}   
\end{frame}
\begin{frame}{Summarizing and Visualizing the Sample Data}
    \begin{block}{Formulas}
    Sample Mean:
      \begin{align}
          \bar{x} = \frac{1}{n}\sum_{i=1}^{n}x_i
      \end{align}
      Sample Variance:
      \begin{align}
          S^2 = \frac{1}{n-1}\sum_{i=1}^{n-1}(x_i-\bar{x})^2
      \end{align}
    \end{block}
\end{frame}
\begin{frame}{Gender-Based comparison}
\begin{figure}[htp]
\centering

\begin{subcolumns}{0.475\columnwidth}
  \begin{subfigure}{0.45\columnwidth}
  \centering
  \vspace{10mm}
  \includegraphics[width=.9\textwidth]{gender sample table.jpeg}
  \caption{}
  \end{subfigure}
\nextsubcolumn
  \begin{subfigure}{0.475\columnwidth}
  \centering
  \includegraphics[width=\textwidth]{male mean, variance.jpeg}
  \caption{}
  \end{subfigure}
\nextsubfigure
  \begin{subfigure}{0.475\columnwidth}
  \centering
  \includegraphics[width=\textwidth]{female mean, variance.jpeg}
  \caption{}
  \end{subfigure}
\end{subcolumns}

\end{figure}
    
\end{frame}
\begin{frame}
    \begin{figure}
     \centering
     \begin{subfigure}[b]{0.475\textwidth}
         \centering
         \includegraphics[width=\textwidth]{wpm gender bar chart.jpeg}
         
         \label{fig:y equals x}
     \end{subfigure}
     \hfill
     \begin{subfigure}[b]{0.475\textwidth}
         \centering
         \includegraphics[width=\textwidth]{wpm gender box plot.jpeg}
         
         \label{fig:three sin x}
     \end{subfigure}
      \hfill
     \begin{subfigure}[b]{0.475\textwidth}
         \centering
         \includegraphics[width=\textwidth]{accuracy gender bar chart.jpeg}
         
         \label{fig:three sin x}
           \end{subfigure}
     \hfill
     \begin{subfigure}[b]{0.475\textwidth}
         \centering
         \includegraphics[width=\textwidth]{accuracy gender box plot.jpeg}
         
         \label{fig:five over x}
     \end{subfigure}
        \caption{}
        \label{fig:three graphs}
\end{figure}
\end{frame}


\begin{frame}{Batch-Based comparison}
\begin{figure}[htp]
\centering

\begin{subcolumns}{0.475\columnwidth}
  \begin{subfigure}{0.45\columnwidth}
  \centering
  \vspace{1.5cm}
  \includegraphics[width=.9\textwidth]{batch mean.jpeg}
  \caption{}
  \end{subfigure}
\nextsubcolumn
  \begin{subfigure}{0.475\columnwidth}
  \centering
  \includegraphics[width=\textwidth]{batch sample table.jpeg}
  \caption{}
  \end{subfigure}
\nextsubfigure
  \begin{subfigure}{0.475\columnwidth}
  \centering
  \includegraphics[width=\textwidth]{batch variance.jpeg}
  \caption{}
  \end{subfigure}
\end{subcolumns}

\end{figure}
    
\end{frame}
\begin{frame}
    \begin{figure}
     \centering
     \begin{subfigure}[b]{0.475\textwidth}
         \centering
         \vspace{10mm}
         \includegraphics[width=\textwidth]{wpm batch bar chart.jpeg}
         
         \label{fig:y equals x}
     \end{subfigure}
     \hfill
     \begin{subfigure}[b]{0.475\textwidth}
         \centering
         \includegraphics[width=\textwidth]{wpm batch box plot.jpeg}
         
         \label{fig:three sin x}
     \end{subfigure}
      \hfill
     \begin{subfigure}[b]{0.475\textwidth}
         \centering
         \includegraphics[width=\textwidth]{accuracy batch bar chart.jpeg}
         
         \label{fig:three sin x}
           \end{subfigure}
     \hfill
     \begin{subfigure}[b]{0.475\textwidth}
         \centering
         \includegraphics[width=\textwidth]{accuracy batch box plot.jpeg}
         
         \label{fig:five over x}
     \end{subfigure}
        \caption{}
        \label{fig:three graphs}
\end{figure}
\end{frame}

\begin{frame}{Branch-Based comparison}
\begin{figure}[htp]
\centering

\begin{subcolumns}{0.475\columnwidth}
  \begin{subfigure}{0.45\columnwidth}
  \centering
  \includegraphics[width=.9\textwidth]{branch wpm mean.jpeg}
  \caption{}
  \end{subfigure}
\nextsubcolumn
  \begin{subfigure}{0.475\columnwidth}
  \centering
  \includegraphics[width=\textwidth]{branch sample table.jpeg}
  \caption{}
  \end{subfigure}
\nextsubfigure
  \begin{subfigure}{0.475\columnwidth}
  \centering
  \includegraphics[width=\textwidth,height=4cm]{wpm group variance.jpeg}
  \caption{}
  \end{subfigure}
\end{subcolumns}

\end{figure}
\end{frame}
\begin{frame}{Branch-Based Comparision}
\begin{figure}[htp]
\centering

\begin{subcolumns}{0.475\columnwidth}
  \begin{subfigure}{0.470\columnwidth}
  \centering
  \includegraphics[width=\textwidth]{branch accuracy variance.jpeg}
  \caption{}
  \end{subfigure}
\nextsubcolumn
  \begin{subfigure}{0.44\columnwidth}
  \centering
  \includegraphics[width=\textwidth]{accuracy group mean.jpeg}
  \caption{}
  \end{subfigure}
\end{subcolumns}

\end{figure}
\end{frame}
\begin{frame}
    \begin{figure}
     \centering
     \begin{subfigure}[b]{0.475\textwidth}
         \centering
         \includegraphics[width=\textwidth]{wpm branch bar chart.jpeg}
         
         \label{fig:y equals x}
     \end{subfigure}
     \hfill
     \begin{subfigure}[b]{0.475\textwidth}
         \centering
         \includegraphics[width=\textwidth]{wpm branch box plot.jpeg} 
         
         \label{fig:three sin x}
     \end{subfigure}
      \hfill
     \begin{subfigure}[b]{0.475\textwidth}
         \centering
         \includegraphics[width=\textwidth]{accuracy branch bar chart.jpeg}
         
         \label{fig:three sin x}
           \end{subfigure}
     \hfill
     \begin{subfigure}[b]{0.475\textwidth}
         \centering
         \includegraphics[width=\textwidth]{accuracy branch box plot .jpeg}
         
         \label{fig:five over x}
     \end{subfigure}
        \caption{}
        \label{fig:three graphs}
\end{figure}
\end{frame}

\begin{frame}
     \section[Analyzing Data]{Summarizing and Visualizing the Sample Data}
\end{frame}
\begin{frame}{CI for estimation of Population Means}
    \begin{block}{Formulas}
      If $X_1, X_2,..., X_n$ are normally distributed with unknown mean $\mu$ and variance $\sigma^2$, then a $(1 - \alpha)100\%$ CI for the population mean $\mu$ is:
    \[ \left(\bar{X} - t_{\alpha/2,n-1}\left(\frac{S}{\sqrt{n}}\right),\ \bar{X} + t_{\alpha/2,n-1}\left(\frac{S}{\sqrt{n}}\right)\right)
    \]
    \end{block}
\end{frame}
\begin{frame}{CI for estimation of Population Means}
\begin{figure}
      \includegraphics[width=\textwidth]{ci pop mean.jpeg} 
\end{figure}
    
\end{frame}
\begin{frame}{CI for estimation of Difference in Population Means}
    \begin{block}{Two sampled pooled t-interval}
        if {$\left(\dfrac{1}{4}<\dfrac{S_X^2}{S_Y^2}<4\right)$}\\
        $X_1,X_2,...,X_n \sim N(\mu_1,\sigma^2)$ and $Y_1,Y_2,...,Y_m \sim N(\mu_2,\sigma^2)$ are independent random samples, then a $(1 - \alpha)100\%$ CI for the difference in the population means, $\mu_1-\mu_2$ is:
        \small\[\left((\bar{X}-\bar{Y})-t_{\alpha/2,n+m-2} S_P\sqrt{\frac{1}{n}+\frac{1}{m}},\  (\bar{X}-\bar{Y})+t_{\alpha/2,n+m-2} S_P\sqrt{\frac{1}{n}+\frac{1}{m}}\right)
        \]
        where 
        \[S_P^2 = \dfrac{(n-1)S_X^2+(m-1)S_Y^2}{n+m-2}
        \]
        \end{block}
\end{frame}
    \begin{frame}{CI for estimation of Difference in Population Means}
        \begin{block}{Welch�s t-interval}
         if \small{$\left(\dfrac{S_X^2}{S_Y^2}<\dfrac{1}{4} \text{\ or\ } 4<\dfrac{S_X^2}{S_Y^2}\right)$}\\ 
        $X_1,X_2,...,X_n \sim N(\mu_1,\sigma_X^2)$ and $Y_1,Y_2,...,Y_m \sim N(\mu_2,\sigma_Y^2)$ are independent random samples, then a $(1 - \alpha)100\%$ CI for the difference in the population means, $\mu_1-\mu_2$ is:
        \small\[\left((\bar{X}-\bar{Y})-t_{\alpha/2,r}\sqrt{\frac{S_X^2}{n}+\frac{S_Y^2}{m}}, \ (\bar{X}-\bar{Y})+t_{\alpha/2,r}\sqrt{\frac{S_X^2}{n}+\frac{S_Y^2}{m}}\right)
        \]
        where 
        \[r = \text{integer part of \ } \dfrac{\left(\frac{S_X^2}{n}+\frac{S_Y^2}{m}\right)^2}{\frac{(S_X^2/n)^2}{n-1} + \frac{(S_Y^2/m)^2}{m-1}}
        \]
        \end{block}
    \end{frame}
        

\begin{frame}{CI for estimation of Difference in Population Means}
\begin{figure}[htp]
\centering

\begin{subcolumns}{0.475\columnwidth}
  \begin{subfigure}{0.45\columnwidth}
  \centering
  \includegraphics[width=.9\textwidth,height=6cm]{branch diff of means.jpeg}
  \caption{}
  \end{subfigure}
\nextsubcolumn
  \begin{subfigure}{0.475\columnwidth}
  \centering
  \includegraphics[width=\textwidth,height=1cm]{gender difference of means.jpeg}
  \caption{}
  \end{subfigure}
\nextsubfigure
  \begin{subfigure}{0.475\columnwidth}
  \centering
  \includegraphics[width=\textwidth,height=2.5cm]{batch diff of means.jpeg}
  \caption{}
  \end{subfigure}
\end{subcolumns}

\end{figure}
   
\end{frame}
\begin{frame}{CI for estimating Population Variance}
    \begin{block}{Formulas}
      If $X_1, X_2,..., X_n$ are normally distributed and $a = \chi^2_{1-\alpha/2,n-1}$, $b = \chi^2_{\alpha/2,n-1}$, then a $(1 - \alpha)100\%$ CI for the population variance $\sigma^2$ is:
    \[ \left(\dfrac{(n-1)S^2}{b}, \dfrac{(n-1)S^2}{a}\right)
    \]
    \end{block}
\end{frame}
\begin{frame}{CI for estimating Population Variance}
\begin{figure}
      \includegraphics[width=\textwidth]{ci pop var.jpeg} 
\end{figure}
\end{frame}
\begin{frame}{CI for estimating ratio of population variance}
    \begin{block}{Formulas}
      If $X_1, X_2,..., X_n \sim N(\mu_X,\sigma_X^2)$ and $Y_1,Y_2,...,Y_m \sim N(\mu_Y,\sigma_Y^2)$ are independent samples and $c = F_{\alpha/2}(m-1, n-1)$, $d = F_{1-\alpha/2}(m-1, n-1)$,
    \[\left(c\dfrac{S_X^2}{S_Y^2},\  d\dfrac{S_X^2}{S_Y^2}\right)
    \]
    \end{block}
\end{frame}
\begin{frame}{CI for estimating ratio of population variance}
\begin{figure}[htp]
\centering

\begin{subcolumns}{0.475\columnwidth}
  \begin{subfigure}{0.45\columnwidth}
  \centering
  \includegraphics[width=.9\textwidth,height=6cm]{branch ci for rov.jpeg}
  \caption{}
  \end{subfigure}
\nextsubcolumn
  \begin{subfigure}{0.475\columnwidth}
  \centering
  \includegraphics[width=\textwidth,height=1cm]{gender ci for rov.jpeg} 
  \caption{}
  \end{subfigure}
\nextsubfigure
  \begin{subfigure}{0.475\columnwidth}
  \centering
  \includegraphics[width=\textwidth,height=2cm]{batch ci for rov.jpeg}
  \caption{}
  \end{subfigure}
\end{subcolumns}

\end{figure}
    
\end{frame}
\begin{frame}{Hypothesis Testing}\label{Hypothesis Testing}
\begin{block}{Definition}
Hypothesis testing is a form of statistical inference that uses data from a sample to draw conclusions about a population parameter or a population probability distribution.
\end{block}
\end{frame}
\begin{frame}{Test for $\mu_1 - \mu_2$: Independent Samples, Unequal Variances}
    \begin{block}{Left-tailed test:}
    \begin{center}
        Null Hypothesis $(H_0)$: $\mu_1$ $\geq$ $\mu_2$\\
        Alternative Hypothesis $(H_a)$: $\mu_1$ $<$ $\mu_2$
\end{center}
The \textbf{Test Statistic (TS)} is:
    \[t' = \frac{\bar{x_1} - \bar{x_2}}{\sqrt{\frac{s_1^2}{n_1}+\frac{s_2^2}{n_2}}}\]
Now,
\[df = \frac{(n_1 - 1)(n_2 - 1)}{(1-c)^2(n_1 - 1) + c^2(n_2 - 1)} \quad \text{where} \quad c = \frac{\frac{s_1^2}{n_1}}{\frac{s_1^2}{n_1}+\frac{s_2^2}{n_2}}
\]

    \end{block}
\end{frame}
\begin{frame}{Test for $\mu_1 - \mu_2$: Independent Samples, Unequal Variances}
  \begin{block}{Left-tailed test contd.}
  df represents degree of freedom. \\

Using p-value approach, we find the p-value for the test statistic (TS) by using t-distribution table and if
\begin{center}
    p-value $\leq$ $\alpha$ : Reject $H_0$ \\
    p-value $ > \alpha$ : Fail to reject $H_0$
\end{center}
  \end{block}  
\end{frame}
\begin{frame}{Test for $\sigma_1^2$ and $\sigma_2^2$}
    \begin{block}{Two-tailed test}
    \begin{center}
        Null Hypothesis $(H_0)$: $\sigma_1^2 = \sigma_2^2$\\
        Alternative Hypothesis $(H_a)$: $\sigma_1^2 \neq \sigma_2^2$ 
\end{center}
The \textbf{Test Statistic (TS)} is:
        \[F = \dfrac{S_1^2}{S_2^2}\]
For a significance level $\alpha$, with $df_1 = n_1-1$ and $df_2 = n_2-1$, if
\begin{center}
    $F\leq F_{1-\alpha/2,df_1,df_2}$ or $F\geq F_{\alpha/2,df_1,df_2}$: Reject $H_0$\\
    $F_{1-\alpha/2,df_1,df_2} < F < F_{\alpha/2,df_1,df_2}$ : Fail to reject $H_0$\\
\end{center}
    \end{block}
\end{frame}

\begin{frame}{Gender-Based Hypothesis Testing}
\begin{figure}[htp]
\centering
  \includegraphics[width=0.85\textwidth]{gender ht .jpeg}
  \caption{}
\end{figure}
\end{frame}

\begin{frame}{Batch-Based Hypothesis Testing }
\begin{figure}[htp]
  \includegraphics[width=1.0\textwidth]{batch based ht.jpeg}
  \caption{}
\end{figure}    
\end{frame}

\begin{frame}{Branch-Based Hypothesis Testing }
\begin{figure}[htp]
  \includegraphics[width=1.0\textwidth]{branch based ht.jpeg}
  \caption{}
\end{figure}
\end{frame}

\begin{frame}
    \section[Conclusion]{Analyzing Data}
\end{frame}
\begin{frame}{Conclusion}
    \textbf{We draw the following conclusions:}
\begin{itemize}
    \item Since we performed an observational study, we can only ascertain correlation between the dependent and independent variables.
    \item In terms of words per minute, males have been found to perform better than their female counterparts. The spread\footnote{By spread, it means the comparison of variances of the groups} of data is not similar in both cases.
    
\end{itemize}


\end{frame}
\begin{frame}{Motivation}
\begin{itemize}
    \item Image to Image translation is a challenging problem that typically requires the development of a specialized model and hand-crafted loss function for the type of translation task being performed.
\item Classical approaches use per-pixel classification or regression models.
\item The problem is that each predicted pixel is independent of the pixels predicted before it and the broader structure of the image might be missed.
\item Ideally, a technique is required that is general, meaning that the same general model and loss function can be used for multiple different image-to-image translation tasks.
\end{itemize}
\end{frame}
\begin{frame}{Pix2Pix Model}
  \begin{figure}[h]
    \centering
    \includegraphics[width=\textwidth]{mesh}
    \caption{\href{https://docs.google.com/spreadsheets/d/1xesjf1nIhxYrOKqsC9ru4PHhpJlf1PWCTVzafvY1Ha4/edit?usp=sharing}{\beamergotobutton{Source}}}
    \label{fig:mesh1}
\end{figure}
\end{frame}

\begin{frame}{Motivation}
\begin{itemize}
\item Traditionally, training an image-to-image translation model requires a dataset comprised of paired examples. 
\item The requirement for a paired training dataset is a limitation as these datasets are challenging and expensive to prepare.
\item Hence, there is a need for techniques for training an image-to-image translation system that does not require paired examples. 
\item We need s setting wherein we can take two collections of unrelated images and extract the general characteristics which can be used for image translation.
\end{itemize}
    
\end{frame}

\begin{frame}{CycleGAN dataflow}
%insert image
https://modelzoo.co/model/mnist-svhn-transfer
\end{frame}
\begin{frame}{Cycle consistency loss}
% insert cycle consistency loss image
  \begin{figure}[h]
    \centering
    \includegraphics[width=\textwidth]{mesh}
    \caption{\href{https://arxiv.org/pdf/1703.10593.pdf}{\beamergotobutton{Source}}}
    \label{fig:mesh1}
\end{figure}

\end{frame}
\begin{frame}{Sample Outputs from the original paper}
% insert cycle consistency loss image
\end{frame}

\begin{frame}
    \section[THANK YOU!]{Conclusion}
    \begin{center}
        \href{https://arxiv.org/pdf/1703.10593.pdf}{\beamergotobutton{Source}}
    \end{center}

\end{frame}
\end{document}
